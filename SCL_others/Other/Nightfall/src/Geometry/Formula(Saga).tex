\begin{small}
\begin{spacing}{0.5}
\begin{multicols*}{2}
	\subsubsection{Heron's Formula}
	\begin{eqnarray*}
		&& S=\sqrt{p(p-a)(p-b)(p-c)} \\
		&& p=\frac{a+b+c}{2}
	\end{eqnarray*}
\subsubsection{四面体内接球球心}
假设$s_i$是第$i$个顶点相对面的面积,则有
\[\left\{
\begin{aligned}
x \ = \ \frac{s_1x_1+s_2x_2+s_3x_3+s_4x_4}{s_1+s_2+s_3+s_4}\\
y \ = \ \frac{s_1y_1+s_2y_2+s_3y_3+s_4y_4}{s_1+s_2+s_3+s_4}\\
z \ = \ \frac{s_1z_1+s_2z_2+s_3z_3+s_4z_4}{s_1+s_2+s_3+s_4}\\
\end{aligned}\right.\]
体积可以使用$1/6$混合积求, 内接球半径为
\[
r \ = \ \frac{3V}{s_1+s_2+s_3+s_4} \\
\]
\subsubsection{三角形内心}
	\[ \frac{a\vec {A} + b\vec{B} + c\vec{C}}{a + b + c} \]
\subsubsection{三角形外心}
	\[ \frac{\vec{A} + \vec{B} - \frac{\overrightarrow {BC} \cdot \overrightarrow{CA}}{\overrightarrow {AB} \times \overrightarrow{BC}}\overrightarrow {AB}^T}{2} \]
\subsubsection{三角形垂心}
	\[ \vec{H} = 3\vec{G} - 2\vec{O} \]
\subsubsection{三角形偏心}
	\[ \frac{-a\vec {A} + b\vec{B} + c\vec{C}}{-a + b + c} \]
	剩余两点的同理. 
\subsubsection{三角形内接外接圆半径}
	\[ r=\frac{2S}{a+b+c},\, R=\frac{abc}{4S} \]
\end{multicols*}
\subsubsection{Pick's Theorem}
\begin{eqnarray*}
S=I+\frac{B}{2}-1
\end{eqnarray*}
$S$ is the area of lattice polygon, $I$ is the number of lattice interior points, and $B$ is the number of lattice boundary points.
\subsubsection{Euler's Formula}
For convex polyhedron: $V-E+F=2$. \\
For planar graph: $|F|=|E|-|V|+n+1$, $n$ denotes the number of connected components.
\subsection{三角公式}
\noindent
\[
\sin(a \pm b) = \sin a \cos b \pm \cos a \sin b
\]
\[
\cos(a \pm b) = \cos a \cos b \mp \sin a \sin b
\]
\[
\tan(a \pm b) = \frac{\tan(a)\pm\tan(b)}{1 \mp \tan(a)\tan(b)}
\]
\[
\tan(a) \pm \tan(b) = \frac{\sin(a \pm b)}{\cos(a)\cos(b)}
\]
\[
\sin(a) + \sin(b) = 2\sin(\frac{a + b}{2})\cos(\frac{a - b}{2})
\]
\[
\sin(a) - \sin(b) = 2\cos(\frac{a + b}{2})\sin(\frac{a - b}{2})
\]
\[
\cos(a) + \cos(b) = 2\cos(\frac{a + b}{2})\cos(\frac{a - b}{2})
\]
\[
\cos(a) - \cos(b) = -2\sin(\frac{a + b}{2})\sin(\frac{a - b}{2})
\]
$
\sin(na) = n\cos^{n-1}a\sin a - \binom{n}{3}\cos^{n-3}a \sin^3a + \binom{n}{5}\cos^{n-5}a\sin^5a - \dots
$
\[
\cos(na) = \cos^{n}a - \binom{n}{2}\cos^{n-2}a \sin^2a + \binom{n}{4}\cos^{n-4}a\sin^4a - \dots
\]
\subsubsection{超球坐标系}
\begin{eqnarray*}
 	x_1 &=& r\cos(\phi_1) \\ 
	x_2 &=& r\sin(\phi_1)\cos(\phi_2) \\
%	x_3 &=& r\sin(\phi_1)\sin(\phi_2)\cos(\phi_3) \\
	\cdots\\
	x_{n-1} &=& r\sin(\phi_1)\cdots\sin(\phi_{n-2})\cos(\phi_{n-1}) \\
	x_n &=& r\sin(\phi_1)\cdots\sin(\phi_{n-2})\sin(\phi_{n-1}) \\
	\phi_{n-1} &\in& [0,2\pi]\\
	\forall {i=1..{n-1}}\phi_i &\in& [0,\pi]\\
\end{eqnarray*}
\subsubsection{三维旋转公式}
绕着$(0,0,0)-(ux,uy,uz)$旋转$\theta$, $(ux,uy,uz)$ 是单位向量
\[
R = \begin{smallmatrix} \cos \theta +u_x^2 \left(1-\cos \theta\right) \quad u_x u_y \left(1-\cos \theta\right) - u_z \sin \theta \quad u_x u_z \left(1-\cos \theta\right) + u_y \sin \theta \\ u_y u_x \left(1-\cos \theta\right) + u_z \sin \theta \quad \cos \theta + u_y^2\left(1-\cos \theta\right) \quad u_y u_z \left(1-\cos \theta\right) - u_x \sin \theta \\ u_z u_x \left(1-\cos \theta\right) - u_y \sin \theta \quad u_z u_y \left(1-\cos \theta\right) + u_x \sin \theta \quad \cos \theta + u_z^2\left(1-\cos \theta\right) 
\end{smallmatrix}.
\]
\[
\begin{bmatrix}
x' \\
y' \\
z' \\
\end{bmatrix} = R
\begin{bmatrix}
x \\
y \\
z \\
\end{bmatrix}
\]
\subsubsection{立体角公式}
\[ \phi \text{ : 二面角}\] 
\[ \Omega = \left(\phi_{ab} + \phi_{bc} + \phi_{ac}\right)\,\mathrm{rad} - \pi\,\mathrm{sr} \]\\
\[\tan \left( \frac{1}{2} \Omega/\mathrm{rad} \right) =
  \frac{\left|\vec a\ \vec b\ \vec c\right|}{abc + \left(\vec a \cdot \vec b\right)c + \left(\vec a \cdot \vec c\right)b + \left(\vec b \cdot \vec c\right)a}\]\\
\[\theta_s = \frac {\theta_a + \theta_b + \theta_c}{2}\]\\
\subsubsection{常用体积公式}
\begin{itemize}
\item Pyramid $V=\frac{1}{3}Sh$.
\item Sphere $V=\frac{4}{3}\pi R^3$.
\item Frustum $V=\frac{1}{3}h(S_1+\sqrt {S_1S_2}+S_2)$.
% \item Ellipsoid \\
% For ellipsoid with the standard equation in a Cartesian coordinate system $\frac{(x-x_0)^2}{a^2}+\frac{(y-y_0)^2}{b^2}+\frac{(z-z_0)^2}{c^2}=1$, $V=\frac{4}{3} \pi abc$.
\item Ellipsoid $V=\frac{4}{3} \pi abc$.
% \item Tetrahedron \\
% For tetrahedron $O-ABC$, let $a=AB,b=BC,c=CA,d=OC,e=OA,f=OB$, ${(12V)}^2=a^2d^2(b^2+c^2+e^2+f^2-a^2-d^2)+b^2e^2(c^2+a^2+f^2+d^2-b^2-e^2)+c^2f^2(a^2+b^2+d^2+e^2-c^2-f^2)-a^2b^2c^2-a^2e^2f^2-d^2b^2f^2-d^2e^2c^2$.
\end{itemize}
\subsubsection{高维球体积}
\begin{eqnarray*}
&& V_2=\pi R^2,\, S_2=2\pi R \\
&& V_3=\frac{4}{3}\pi R^3,\, S_3=4\pi R^2 \\
&& V_4=\frac{1}{2}\pi ^2 R^4,\, S_4=2\pi ^2 R^3 \\
%&& V_5=\frac{8}{15}\pi ^2 R^5,\, S_5=\frac{8}{3}\pi ^2 R^4 \\
%&& V_6=\frac{1}{6}\pi ^3 R^6,\, S_6=\pi ^3 R^5 \\
&& \mathrm{Generally}, V_n=\frac{2\pi}{n}V_{n-2},\, S_{n-1}=\frac{2\pi}{n-2}S_{n-3} \\
&& \mathrm{Where}, S_0=2,\, V_1=2,\, S_1=2\pi ,\, V_2=\pi
\end{eqnarray*}
\end{spacing}
\end{small}